 \documentclass[]{report}
%\documentclass[10pt,a4paper,headinclude=true,twoside]{report}
\usepackage[latin1]{inputenc}
\usepackage[a4paper]{geometry}
\usepackage{a4wide}
\usepackage{amsmath}
\usepackage{amsfonts}
\usepackage{amssymb}
\usepackage{graphicx}
\usepackage{hyperref}
\usepackage{pdflscape} % dlia landscape orientation 
\hypersetup{colorlinks,citecolor=black,filecolor=black,linkcolor=black,urlcolor=black}
\usepackage{float}
\usepackage{setspace}
\usepackage{titlesec}


\begin{document}
%\onehalfspacing
\title{CS31310: Process for project}
\author{Edgar Ivanov\\ edi@aber.ac.uk \\ Department of Computer Science, Aberystwyth University}
\date{\today}
\maketitle
\section*{Project outline}
My final year project will have both hardware and software parts. The aim of the project is to build additional unit which will ensure that camera mounted on the rover is always in stable position. Currently there is a Pan-and-Tilt Unit (PTU) which controls the camera and ensures that it is in a stable position. It uses gyroscopes to acquire current position in the space and based on that adjusts position of the camera. However gyroscopes are not perfect and tend to drift over the time, as well as provide wrong data at the different temperature. 

I will be building additional unit which will acquire current position using the more accurate accelerometer and then will provide PTU with the calibration data.

\section*{Process}
At this point it is difficult for me to choose the right development methodology for my project. There are a few of them and I have very limited experience of using any of them. Since I didn't go in too much details of my project yet, it will be a bit of guess work trying to find suitable methodology.

Essentially my customer will be my project supervisory, because I am building system for him and he has deep interest in this project being completed indeed. There will be no big team of developers working on the project, just me with some help from my supervisor. The final goal of the project is quite clear, so it should be a matter of gathering requirements and conducting a proper planning.

On the other hand, I have a very limited knowledge of C programming language and experience of working with proposed hardware, so for me it will be an exploratory work to find out how to fit all parts together and make them work. I may not be able to predict all the pitfalls on the course of the project and as such may need to revise and alter my plan. Having said that I need a flexible methodology which allows to go back and change requirements together with the design.  

1 key steps of the project and how you will use these on single person project
2 entire lifecycle for your project
3 Issues: Key steps of the project
		1 establishing requirements
		2 tracking progress
		3 create a design
		4 undertake development and testing
		
		
		
 1 It is likely that I will be coming back to the requirements analysis during development so phased development would be unsuitable
 2 Difference between methodologies and life-cycle 
\end{document}