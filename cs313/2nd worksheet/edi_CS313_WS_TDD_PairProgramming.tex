\documentclass[]{report}
%\documentclass[11pt,a4paper,headinclude=true,twoside]{report}
\usepackage[latin1]{inputenc}
\usepackage[a4paper]{geometry}
\usepackage{a4wide}
\usepackage{amsmath}
\usepackage{amsfonts}
\usepackage{amssymb}
\usepackage{graphicx}
\usepackage{hyperref}
\usepackage{pdflscape} % dlia landscape orientation 
\hypersetup{colorlinks,citecolor=black,filecolor=black,linkcolor=black,urlcolor=black}
\usepackage{float}
\usepackage{setspace}
\usepackage{titlesec}


\begin{document}
%\onehalfspacing
\title{CS31310: Test Driven Development and Pair Programming Session Reflection}
\author{Edgar Ivanov\\ edi@aber.ac.uk \\ Department of Computer Science, Aberystwyth University}
\date{\today}
\maketitle
We have recently had a workshop session; we were asked to write some code in pairs using TDD technique. First half of the session I worked with my friend whose skills in Java where a bit lower that mine, this fact influenced our decision when picking up roles. We decided that I will take the navigator role and my friend will be in a driver role. We then discusses how our application may look like and what we could write tests for. The simplest thing we could think of was to test creation of the fruit objects (banana, apple, orange) and the default price value assigned to them during the creation. While my friend was writing tests I was following on the screen everything that was typed in to make sure that there are no mistakes in the code and that he follows our plan.  At the beginning it was a bit difficult to concentrate on writing tests since it was tempting to write code that was required for the implementation first and then write actual tests. After we finished coding tests we run them to make sure that they actually fail. Afterwards we coded simplest implementations for our fruit classes that would allow tests to pass. While coding tests I thought that there is probably no point in testing such simple behaviour and we should move straight to adding objects to the list and then testing if total value of all fruits in cart matches the expected one, but after all it was about getting overall idea and the experience. While writing tests we where continuously sharing thoughts of what else could be tested, it was helpful because both of us come up with a few good ideas and throughout discussion we rounded them up. 


With regards to the "red-green-refactor" approach we used only first two. There was no need to refactor anything in the fruit classes because they were as simple as they could be anyway just returning value of the variable, although later on we had to move some functionality to the superclass to make code management easier. 


After we changed pairs I explained my new partner what our code was doing and what tests we implemented so far. As I found out later my new partner was much more efficient in Java than me. He was a bit quicker in making design decisions and I was asking questions with regards to the them. I also noticed that when I didn't understand some aspects I started loosing interest in what is being done.

In overall it was interesting session, but I  have some contradictory feelings about TDD. It seems to be a waist of time testing trivial things like a creation on of the new object. I don't think there is any point writing tests for each and every method in the class, like we did in the workshop session. I also don't think that TDD technique is suitable for the small projects with a few classes where testing can be done by simply running application. 

Pair programming also seems to be good when both participants have equal amount of the knowledge and quickly catch up with what is going on. But if one of them has more knowledge that the other then he needs to explain what is being done to his colleague, otherwise there is no point in carrying work together since partner will not be able to understand code and provide help when issue arise.  

\bibliographystyle{ieeetr}
\bibliography{bibl}
\end{document}