\documentclass[]{report}
%\documentclass[11pt,a4paper,headinclude=true,twoside]{report}
\usepackage[latin1]{inputenc}
\usepackage[a4paper]{geometry}
\usepackage{a4wide}
\usepackage{amsmath}
\usepackage{amsfonts}
\usepackage{amssymb}
\usepackage{graphicx}
\usepackage{hyperref}
\usepackage{pdflscape} % dlia landscape orientation 
\hypersetup{colorlinks,citecolor=black,filecolor=black,linkcolor=black,urlcolor=black}
\usepackage{float}
\usepackage{setspace}
\usepackage{titlesec}


\begin{document}
%\onehalfspacing
\title{CS31310: Test Driven Development and Pair Programming Session Reflection}
\author{Edgar Ivanov\\ edi@aber.ac.uk \\ Department of Computer Science, Aberystwyth University}
\date{\today}
\maketitle
o  How	did	you	interact	with	the	people	that	you	worked	with?	Was	the	
interaction	helpful?	Explain	why	you	thought	it	was	or	wasn?t	helpful.	
o  Comment	on	the	use	of	the	?red-green-refactor?	approach	during	the	exercise.	
Was	this	a	help	or	a	hindrance	to	developing	the	code?	Briefly	discuss	your	
thoughts	on	this.	
o  Any	other	points	that	you	would	like	to	raise	about	the	exercise.		

We had a workshop session where we needed to develop some code while working in pairs and using TDD technique. First half of the session I worked with my friend whose skills in Java where a bit lower that mine. At the beginning we decided that I will take the navigator role and my friend will be in a driver role. We then discusses how our application may look like and what we could write tests for. The simplest thing we could think of was to test creation of the fruit object and default price value assigned to it during the creation. While my friend was writing tests I was following on the screen everything that was typed in to make sure that there are no mistakes in the code and that he follows our plan.  At the beginning it was a bit difficult to concentrate on writing tests since was tempting to write code that was required for the implementation first and then write actual tests. After we run test and they failed we then coded simplest implementations for our fruit classes that would allow tests to pass. While coding tests I actually thought that there is probably no point in testing such simple behaviour and we should move straight to adding objects to the list and then testing if total value matches with expected one, but after all it was about getting overall idea and the experience. While writing tests we where continuously sharing thoughts of what else we could test, it was helpful because both of us come up with a few good ideas. In red-green-refactor approach we used only first two, there was actually no need to refactor anything in the fruit class because it was as simple as it could be anyway.        

\bibliographystyle{ieeetr}
\bibliography{bibl}
\end{document}